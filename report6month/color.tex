%%% LaTeX Template: Article/Thesis/etc. with colored headings and special fonts
%%%
%%% Source: http://www.howtotex.com/
%%% Feel free to distribute this template, but please keep to referal to http://www.howtotex.com/ here.
%%% February 2011

%%%%% Preamble
\documentclass[10pt,a4paper]{article}

\usepackage[T1]{fontenc}
\usepackage[bitstream-charter]{mathdesign}

\usepackage[latin1]{inputenc}							% Input encoding
\usepackage{amsmath}									% Math

\usepackage{xcolor}
\definecolor{bl}{rgb}{0.0,0.2,0.6} 


\usepackage{sectsty}
\usepackage[compact]{titlesec} 
\allsectionsfont{\color{bl}\scshape\selectfont}

%%%%% Definitions
% Define a new command that prints the title only
\makeatletter							% Begin definition
\def\printtitle{%						% Define command: \printtitle
    {\color{bl} \centering \huge \sc \textbf{\@title}\par}}		% Typesetting
\makeatother							% End definition

\title{AUTOMATION \\ 
		\large \vspace*{-10pt}The preserving memories\vspace*{10pt}}

% Define a new command that prints the author(s) only
\makeatletter							% Begin definition
\def\printauthor{%					% Define command: \printauthor
    {\centering \small \@author}}				% Typesetting
\makeatother							% End definition

\author{%
	Sandeep Kaur \\
        G.N.D.E.C Ludhiana\\
        D3 IT \\
        96011(90370821211) \\
	mkaurkhalsa@gmail.com \\
	\vspace{20pt}
	}

% Custom headers and footers
\usepackage{fancyhdr}
	\pagestyle{fancy}					% Enabling the custom headers/footers
\usepackage{lastpage}	
	% Header (empty)
	\lhead{}
	\chead{}
	\rhead{}
	% Footer (you may change this to your own needs)
	\lfoot{\footnotesize \texttt{HowToTeX.com} - Template}
	\cfoot{}
	\rfoot{\footnotesize page \thepage\ of \pageref{LastPage}}	% "Page 1 of 2"
	\renewcommand{\headrulewidth}{0.0pt}
	\renewcommand{\footrulewidth}{0.4pt}

% Change the abstract environment
\usepackage[runin]{abstract}			% runin option for a run-in title
\setlength\absleftindent{30pt}		% left margin
\setlength\absrightindent{30pt}		% right margin
\abslabeldelim{\quad}						% 
\setlength{\abstitleskip}{-10pt}
\renewcommand{\abstractname}{}
\renewcommand{\abstracttextfont}{\color{bl} \small \slshape}	% slanted text


%%% Start of the document
\begin{document}
%%% Top of the page: Author, Title and Abstact
\printtitle 

\printauthor

\begin{abstract}
An abstract is a brief summary of a research article, thesis, review, conference proceeding or any in-depth analysis of a particular subject or discipline, and is often used to help the reader quickly ascertain the paper's purpose. When used, an abstract always appears at the beginning of a manuscript, acting as the point of entry for any given scientific paper or patent application. Abstracting and indexing services for various academic disciplines are aimed at compiling a body of literature for that particular subject.
\end{abstract}

%%% Start of the 'real' content of the article, using a two column layout
\section{Heading on level 1 (section)}
Lorem ipsum dolor sit amet, consectetuer adipiscing elit. Aenean commodo ligula eget dolor. Aenean massa. Cum sociis natoque penatibus et magnis dis parturient montes, nascetur ridiculus mus. Donec quam felis, ultricies nec, pellentesque eu, pretium quis, sem. Nulla consequat massa quis enim. Donec pede justo, fringilla vel, aliquet nec, vulputate eget, arcu. In enim justo, rhoncus ut, imperdiet a, venenatis vitae, justo. 
\begin{align} 
	\begin{split}
	(x+y)^3 	&= (x+y)^2(x+y)\\
					&=(x^2+2xy+y^2)(x+y)\\
					&=(x^3+2x^2y+xy^2) + (x^2y+2xy^2+y^3)\\
					&=x^3+3x^2y+3xy^2+y^3
	\end{split}					
\end{align}
Nullam dictum felis eu pede mollis pretium. Integer tincidunt. Cras dapibus. Vivamus elementum semper nisi. Aenean vulputate eleifend tellus. Aenean leo ligula, porttitor eu, consequat vitae, eleifend ac, enim. Aliquam lorem ante, dapibus in, viverra quis, feugiat a, tellus. Phasellus viverra nulla ut metus varius laoreet. Quisque rutrum. Aenean imperdiet. Etiam ultricies nisi vel augue. Curabitur ullamcorper ultricies.

\subsection{Heading on level 2 (subsection)}
Lorem ipsum dolor sit amet, consectetuer adipiscing elit. Aenean commodo ligula eget dolor. Aenean massa. Cum sociis natoque penatibus et magnis dis parturient montes, nascetur ridiculus mus. Donec quam felis, ultricies nec, pellentesque eu, pretium quis, sem. 
\begin{align}
	A = 
	\begin{bmatrix}
	A_{11} & A_{21} \\
  	A_{21} & A_{22}
	\end{bmatrix}
\end{align}

\subsubsection{Heading on level 3 (subsubsection)}
Nulla consequat massa quis enim. Donec pede justo, fringilla vel, aliquet nec, vulputate eget, arcu. In enim justo, rhoncus ut, imperdiet a, venenatis vitae, justo. Nullam dictum felis eu pede mollis pretium. Integer tincidunt. Cras dapibus. Vivamus elementum semper nisi. Aenean vulputate eleifend tellus. Aenean leo ligula, porttitor eu, consequat vitae, eleifend ac, enim.

\paragraph{Heading on level 4 (paragraph)}
Lorem ipsum dolor sit amet, consectetuer adipiscing elit. Aenean commodo ligula eget dolor. Aenean massa. Cum sociis natoque penatibus et magnis dis parturient montes, nascetur ridiculus mus. Donec quam felis, ultricies nec, pellentesque eu, pretium quis, sem. Nulla consequat massa quis enim. 


\section{Lists}
\subsection{Example for list (itemize)}
\begin{itemize}
	\item First item in a list 
	\item Second item in a list 
	\item Third item in a list
\end{itemize}

\subsubsection{Example for list (3*itemize)}
\begin{itemize}
	\item First item in a list 
		\begin{itemize}
		\item First item in a list 
			\begin{itemize}
			\item First item in a list 
			\item Second item in a list 
			\end{itemize}
		\item Second item in a list 
		\end{itemize}
	\item Second item in a list 
\end{itemize}

\subsection{Example for list (enumerate)}
\begin{enumerate}
	\item First item in a list 
	\item Second item in a list 
	\item Third item in a list
\end{enumerate}

\subsubsection{Example for list (3*enumerate)}
\begin{enumerate}
	\item First item in a list 
		\begin{enumerate}
		\item First item in a list 
			\begin{enumerate}
			\item First item in a list 
			\item Second item in a list 
			\end{enumerate}
		\item Second item in a list 
		\end{enumerate}
	\item Second item in a list 
\end{enumerate}

\end{document}
